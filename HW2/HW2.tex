\documentclass[12pt]{article}
\title{PHIL 322---Modal Logic \\Homework 2}
\author{Shawn Wu}
\date{October 18, 2022}
%Add your name and the date in the brackets above, and then remove the percentage 
%sign from the beginning of the line. 
%this just makes sure that all the standard math fonts/definitions/modes can be 
%used.
\usepackage{amsfonts, amsmath, amssymb, amsthm, textcomp}
%using fullpage does what you think it does, forces the full page to be used rather
%than the LaTeX norm, which is a reasonably narrow column.
\usepackage{fullpage}
%Everything up to here was just setting things up. Everything that you will 
%actually see occurs between the \begin{document} and \end{document} tags
\begin{document}
\maketitle
%This gives your name, the date, and the rest of the information you might have entered at the top of the file. 
\noindent
\begin{center}
    ------(1)------
\end{center}
\begin{proof} $ $
    \textbf{Partially Functional}
\end{proof}

\newpage
\noindent
\begin{center}
    ------(2a)------
\end{center}
\begin{proof}
    
\end{proof}

\noindent
\begin{center}
    ------(2b)------
\end{center}
\begin{proof}
    
\end{proof}

\newpage
\noindent
\begin{center}
    ------(3a)------
\end{center} 
\begin{proof}
    
\end{proof}

\noindent
\begin{center}
    ------(3b)------
\end{center} 
\begin{proof}
    
\end{proof}


\newpage
\noindent
\begin{center}
    ------(4)------
\end{center} 
\begin{proof}
    
\end{proof}


\newpage
\noindent
\begin{center}
    ------(5a)------
\end{center} 
\begin{proof}
    
\end{proof}

\noindent
\begin{center}
    ------(5b)------
\end{center} 
\begin{proof}
    
\end{proof}

\noindent
\begin{center}
    ------(5c)------
\end{center} 
\begin{proof}
    
\end{proof}

\noindent
\begin{center}
    ------(5d)------
\end{center} 
\begin{proof}
    
\end{proof}

\noindent
\begin{center}
    ------(5e)------
\end{center} 
\begin{proof}
    
\end{proof}
\end{document}