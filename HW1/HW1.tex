\documentclass[11pt]{article}
\title{PHIL 322---Modal Logic \\Homework 1}
\author{}
%\date{}
%Add your name and the date in the brackets above, and then remove the percentage 
%sign from the beginning of the line. 
%this just makes sure that all the standard math fonts/definitions/modes can be 
%used.
\usepackage{amsfonts, amsmath, amssymb, amsthm}
%using fullpage does what you think it does, forces the full page to be used rather
%than the LaTeX norm, which is a reasonably narrow column.
\usepackage{fullpage}
%these commands allow you to declare theorems/lemmas/corollaries/definitions. They 
%will all be numbered sequentially. That is, your document will remember what the 
%last theorem number used was, and, the next theorem you write will have the 
%subsequent number. As it is set up here, theorems, lemmas, etc. all share the same 
%numbering sequence. That means, if you have a theorem and then a corollary, the 
%theorem will be numbered as 1 as the corollary as 2.
\newtheorem{theorem}{Theorem}[section]
\newtheorem{corollary}[theorem]{Corollary}
\newtheorem{lemma}[theorem]{Lemma}
\newtheorem{definition}[theorem]{Definition}
%To declare a theorem you can write:
%
%\begin{theorem}
%There are infinitely many prime numbers.
%\end{theorem}
%
%prime $p$. Then consider the number we get by taking 
%
%$$n=(2 \times 3 \times 5 \times 7 \times 11 \dots \times p) +1$$
%
%Clearly, $n$ cannot be prime, by our assumption. Therefore, it is composite. But 
%the fundamental theorem of arithmetic tells us that every positive integer has a 
(unique) prime factorization. This means that $n$ is evenly divided by some prime 
number $q$. But notice that $q$ cannot be any of the primes between $2$ and $p$, 
because these will all have a remainder of $1$. Therefore, $n$ must be a prime 
factor greater than $p$. This is a contradiction.
%
%\end{proof}
%Everything up to here was just setting things up. Everything that you will 
%actually see occurs between the \begin{document} and \end{document} tags
\begin{document}
\maketitle
%This gives your name, the date, and the rest of the information you might have 
entered at the top of the file. 
\noindent
1. Call a binary relation $R$ on a set $S$ \emph{asymmetric} if for no $a,b \in S$ 
both $\langle a, b \rangle \in R$ and $\langle b, a \rangle \in R$. Specify a 
relation $R$ and a set $S$ such that $R$ is a relation on $S$ and $R$ is reflexive 
and asymmetric. 
\newpage
\noindent
2. Prove that the transitive closure, $R^+$ of a binary relation $R$ is actually 
transitive. You can do this however you like. Here is one suggestion:
\begin{enumerate}
\renewcommand{\labelenumi}{\alph{enumi}.}
\item Prove that, where $X$ is a set and $R \subseteq X^2$ (i.e.,\ it is a binary 
relation on $X$), then $\langle a,b \rangle \in R^n$ (as defined on page 191 of 
your text) iff there is a path of length $n$ between $a$ and $b$ in the graph of 
$R$. 
\item Use this result to then argue that if $\langle a,b\rangle \in R^+$ and $\
langle b,c\rangle \in R^+$, then $\langle a,c\rangle \in R^+$.
\end{enumerate}
\newpage
\noindent
3. Let $R \subseteq X^2$ be a binary relation on $X$ that is antisymmetric, 
transitive and total/connex (notice that totality implies reflexivity). Show that 
$$R' := \{\langle x,y \rangle \in X^2: \langle y, x \rangle \notin R\}$$
\noindent
is irreflexive, transitive, and connected/semi-connex. (A relation on $X$  is \
emph{connected}, or \emph{semi-connex}, if, for all $x,y \in X$, if $x\neq y$ then 
either $\langle x, y \rangle \in R$ or $\langle y, x \rangle \in R$. A relation 
is \emph{connex}, or \emph{total}, when, for all $x,y \in X$, either $\langle x, 
y \rangle \in R$ or $\langle y, x \rangle \in R$.)
\newpage
\noindent
4. Call a relation $R$ on a set $S$ \emph{universal} when, for all $x, y \in S$, $\
langle x, y \rangle \in R$. Let $R$ be an equivalence relation and define, for each
$x \in S$, the set 
$$[x] = \{y \in S : \langle x , y \rangle \in R\}$$
\noindent
Call this the \emph{equivalence class} of $x$ (with respect to $R$). Prove that all
of the following are true:
\begin{enumerate}
\renewcommand{\labelenumi}{\alph{enumi}.}
\item $x \in [x]$, for all $x \in S$;
\item $R$ is universal on each equivalence class $[x]$;
\item Every element of $S$ is in one and only one equivalence class.
\end{enumerate}
\newpage
\noindent
5. Prove that the set $\{\neg, \land, \lor\}$ is truth-functionally complete. Hint:
Consider the following truth table for some arbitrary binary connective $\circ$:
\medskip
\begin{center}
\begin{tabular}{cc|c}
$\varphi$ & $\psi$ & $\varphi \circ \psi$ \\ \hline
$\mathsf{t}$ & $\mathsf{t}$ & $\mathsf{v}_1$ \\
$\mathsf{t}$ & $\mathsf{f}$ & $\mathsf{v}_2$ \\
$\mathsf{f}$ & $\mathsf{t}$ & $\mathsf{v}_3$ \\
$\mathsf{f}$ & $\mathsf{f}$ & $\mathsf{v}_4$ \\
\end{tabular}
\end{center}
\medskip
\noindent
where $\mathsf{v}_i$ is the truth value for that row. There are at least two 
options now.  One is a brute force argument that is effective, but not so exciting.
See if you can come up with a different one. Start by thinking about what you could
do if all the values are $\mathsf{f}$. What would express such a function? Then 
think about what would happen if one of the values were $\mathsf{t}$. How might you
express that? Then go from there\dots
\end{document}