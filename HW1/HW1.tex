\documentclass[12pt]{article}
\title{PHIL 322---Modal Logic \\Homework 1}
\author{Shawn Wu}
\date{September 27, 2022}
%Add your name and the date in the brackets above, and then remove the percentage 
%sign from the beginning of the line. 
%this just makes sure that all the standard math fonts/definitions/modes can be 
%used.
\usepackage{amsfonts, amsmath, amssymb, amsthm, textcomp}
%using fullpage does what you think it does, forces the full page to be used rather
%than the LaTeX norm, which is a reasonably narrow column.
\usepackage{fullpage}
%Everything up to here was just setting things up. Everything that you will 
%actually see occurs between the \begin{document} and \end{document} tags
\begin{document}
\maketitle
%This gives your name, the date, and the rest of the information you might have entered at the top of the file. 
\noindent
1. Call a binary relation $R$ on a set $S$ \emph{asymmetric} if for no $a,b \in S$ 
both $\langle a, b \rangle \in R$ and $\langle b, a \rangle \in R$. Specify a 
relation $R$ and a set $S$ such that $R$ is a relation on $S$ and $R$ is reflexive 
and asymmetric.
\begin{center}
    $\ast$~$\ast$~$\ast$
\end{center}
Let $S = \varnothing$ and $R = \varnothing$. $R$ is both relexive and asymmetric.
\begin{proof}
Note that a relation $R \subseteq S^2$ is relexive iff $\forall x \in S, \langle x ,x \rangle \in R.$ 
Since $S$ is an empty set, there are no elements in $S$. This makes the above universally quantified statement vacously ture.
Hence, $R$ is relexive.
\\
Also note that a relation $R \subseteq S^2$ is asymmetric if for no pair $x,y \in S$ we have both $Rxy$ and $Ryx.$
Since $S$ is empty, then we couldn't find any pairs of $x,y \in S$ that dissatisfy the mentioned properties.
Hence, $R$ is asymmetric. 
\end{proof}

\newpage
\noindent
2. Prove that the transitive closure, $R^+$ of a binary relation $R$ is actually 
transitive. You can do this however you like. Here is one suggestion:
\begin{enumerate}
\renewcommand{\labelenumi}{\alph{enumi}.}
\item Prove that, where $X$ is a set and $R \subseteq X^2$ (i.e.,\ it is a binary 
relation on $X$), then $\langle a,b \rangle \in R^n$ (as defined on page 191 of 
your text) iff there is a path of length $n$ between $a$ and $b$ in the graph of 
$R$. 
\item Use this result to then argue that if $\langle a,b\rangle \in R^+$ and 
$\langle b,c\rangle \in R^+$, then $\langle a,c\rangle \in R^+$.
\end{enumerate}
\begin{center}
    $\ast$~$\ast$~$\ast$
\end{center}
\begin{proof}
    
\end{proof}

\newpage
\noindent
3. Let $R \subseteq X^2$ be a binary relation on $X$ that is antisymmetric, 
transitive and total/connex (notice that totality implies reflexivity). Show that 
$$R' := \{\langle x,y \rangle \in X^2: \langle y, x \rangle \notin R\}$$
\noindent
is irreflexive, transitive, and connected/semi-connex. (A relation on $X$  is \emph{connected}, 
or \emph{semi-connex}, if, for all $x,y \in X$, if $x\neq y$ then 
either $\langle x, y \rangle \in R$ or $\langle y, x \rangle \in R$. A relation 
is \emph{connex}, or \emph{total}, when, for all $x,y \in X$, either $\langle x, 
y \rangle \in R$ or $\langle y, x \rangle \in R$.)
\begin{center}
    $\ast$~$\ast$~$\ast$
\end{center}
\begin{proof}
    First note that either $X = \varnothing$ or $X \neq \varnothing$.\\

    \noindent\textbf{CASE 1}($X = \varnothing$): If $X = \varnothing$, then $\varnothing = R \subseteq X^2 = \varnothing$. 
    Observe that the assumption on $R$ being antisymmetric, transitive and connex still holds.\\
    
    \indent 1) $R$ is vacously antisymmetric. Note that $R \subseteq X^2$ is antisymmetric iff
    $\forall x, y \in X, ([Rxy \wedge Ryx] \implies x = y)$. Therefore, $R$ is \emph{not} antisymmetric 
    iff $\exists x, y \in X, (Rxy \wedge Ryx \wedge x \neq y)$. Since $R$ is empty, then
    we can't find a pair of $x, y$ in $R$ that satisfy the requirement. $R$ is not \emph{not} 
    antisymmetric, hence antisymmetric.\\

    \indent 2) $R$ is vacously transitive. Note that $R$ is \emph{not} transitive iff 
    $\exists x,y,z \in X, (Rxy \wedge Ryz \wedge \neg Rxz)$. Since $R$ is empty, then we can't find
    the corresponding pairs in $x,y,z$ that can satisfy the requirement. 
    $R$ is not \emph{not} transitive, hence transitive.\\
    
    \indent 3) $R$ is vacously connex. Note that $R$ is \emph{not} connex iff
    $\exists x, y \in X, (\neg Rxy \wedge \neg Ryx)$. Since $X$ is empty, we can't find the
    corresponding pairs of $x, y$ that can satisfy the requirement. $R$ is not \emph{not} 
    connex, hence connex.\\

    \noindent We now show that $R'$ is irreflexive, transitive, and connected.
    First note that $\varnothing = R' = \{\langle x,y \rangle \in X^2 = \varnothing: \langle y, x \rangle \notin R\}$.\\

    \indent 1) $R'$ is irreflexive. 

\end{proof}

\newpage
\noindent
4. Call a relation $R$ on a set $S$ \emph{universal} when, for all $x, y \in S$, 
$\langle x, y \rangle \in R$. Let $R$ be an equivalence relation and define, for each
$x \in S$, the set 
$$[x] = \{y \in S : \langle x , y \rangle \in R\}$$
\noindent
Call this the \emph{equivalence class} of $x$ (with respect to $R$). Prove that all
of the following are true:
\begin{enumerate}
\renewcommand{\labelenumi}{\alph{enumi}.}
\item $x \in [x]$, for all $x \in S$;
\item $R$ is universal on each equivalence class $[x]$;
\item Every element of $S$ is in one and only one equivalence class.
\end{enumerate}
~~~
\begin{center}
    ------(a)------
\end{center}
\begin{proof}
    Pick any $x \in S$ and let $[x]_R$ be the equivalence class defined above. 
    Given $R$ is an equivalence relation, we have $\langle x,x \rangle \in R$ as $R$ is relexive.
    Hence, $x \in [x]_R$.
\end{proof}
~~~
\begin{center}
    ------(b)------
\end{center}
\begin{proof}
    Suppose we have a equivalence class $[x]_R$ as defined above. And suppose $p, q \in [x]_R$.
    Since $p \in [x]_R,$ we have $\langle x, p \rangle \in R$. Since $q \in [x]_R$, we also have
    $\langle x,q \rangle \in R$. Note that $R$ is symmetric as it is an equivalence relation;
    therefore, we have $\langle p,x \rangle \in R$. Also note that $R$ is transitive as it is an
    equivalence relation; therefore, $\langle p,q \rangle \in R$. Hence, $\forall x_1, x_2 \in [x]_R, 
    \langle x_1, x_2 \rangle \in R$; $R$ is univeral on this equivalence class. Since we picked a
    random equivalence class, $R$ is univeral on \emph{each} $[x]_R$.
\end{proof}
~~~
\begin{center}
    ------(c)------
\end{center}
\begin{proof}
    Pick any $x \in S$. From $(a)$ we know that element $x$ is in at least one equivalence class, 
    namely $[x]_R$. We now show that $[x]_R$ is the only equivalence class with respect to 
    $R$ that $x$ is in.\\
    We know that $x \in [x]_R$ from $(a)$. But now suppose that $x \in [\bar{x}]_R$ with
    $\bar{x} \in S$. We show that $[x]_R = [\bar{x}]_R$.\\
    Pick any $s \in [x]_R$, then we know that $\langle x,s \rangle \in R$ according to the
    definition of equivalence class. Similarly, since $x \in [\bar{x}]_R$, we have 
    $\langle \bar{x}, x \rangle \in R$. Thus, we have $\langle \bar{x}, s \rangle \in R$ 
    as $R$ is transitive.
    Hence, $s \in [\bar{x}]_R$. We have $[x]_R \subseteq [\bar{x}]_R$.\\
    Now pick any $s \in [\bar{x}]_R$. From (b) we know that $R$ is univeral on $[\bar{x}]_R$ and 
    $s, x \in [\bar{x}]_R$. Thus, $\langle x, s \rangle \in R$, which means that 
    $s \in [x]_R$. We have $[x]_R \supseteq [\bar{x}]_R$. \\
    Therefore, $[x]_R = [\bar{x}]_R$. This means that any other equivalence classes $x$ is in
    are just $[x]_R$. Hence, every element of $S$ is in one and only one equivalence class.

\end{proof}

\newpage
\noindent
5. Prove that the set $\{\neg, \land, \lor\}$ is truth-functionally complete. Hint:
Consider the following truth table for some arbitrary binary connective $\circ$:
\medskip
\begin{center}
\begin{tabular}{cc|c}
$\varphi$ & $\psi$ & $\varphi \circ \psi$ \\ \hline
$\mathsf{t}$ & $\mathsf{t}$ & $\mathsf{v}_1$ \\
$\mathsf{t}$ & $\mathsf{f}$ & $\mathsf{v}_2$ \\
$\mathsf{f}$ & $\mathsf{t}$ & $\mathsf{v}_3$ \\
$\mathsf{f}$ & $\mathsf{f}$ & $\mathsf{v}_4$ \\
\end{tabular}
\end{center}
\medskip
\noindent
where $\mathsf{v}_i$ is the truth value for that row. There are at least two 
options now.  One is a brute force argument that is effective, but not so exciting.
See if you can come up with a different one. Start by thinking about what you could
do if all the values are $\mathsf{f}$. What would express such a function? Then 
think about what would happen if one of the values were $\mathsf{t}$. How might you
express that? Then go from there\dots
\begin{center}
    $\ast$~$\ast$~$\ast$
\end{center}
\begin{proof}
    
\end{proof}
\end{document}