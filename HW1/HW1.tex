\documentclass[12pt]{article}
\title{PHIL 322---Modal Logic \\Homework 1}
\author{Shawn Wu}
\date{September 27, 2022}
%Add your name and the date in the brackets above, and then remove the percentage 
%sign from the beginning of the line. 
%this just makes sure that all the standard math fonts/definitions/modes can be 
%used.
\usepackage{amsfonts, amsmath, amssymb, amsthm, textcomp}
%using fullpage does what you think it does, forces the full page to be used rather
%than the LaTeX norm, which is a reasonably narrow column.
\usepackage{fullpage}
%Everything up to here was just setting things up. Everything that you will 
%actually see occurs between the \begin{document} and \end{document} tags
\begin{document}
\maketitle
%This gives your name, the date, and the rest of the information you might have entered at the top of the file. 
\noindent
1. Call a binary relation $R$ on a set $S$ \emph{asymmetric} if for no $a,b \in S$ 
both $\langle a, b \rangle \in R$ and $\langle b, a \rangle \in R$. Specify a 
relation $R$ and a set $S$ such that $R$ is a relation on $S$ and $R$ is reflexive 
and asymmetric.
\begin{center}
    $\ast$~$\ast$~$\ast$
\end{center}
Let $S = \varnothing$ and $R = \varnothing$. $R$ is both relexive and asymmetric.
\begin{proof}
Note that a relation $R \subseteq S^2$ is relexive iff $\forall x \in S, Rxx.$
This means that $R$ is \emph{not} reflexive iff $\exists x \in S, \neg Rxx$.
Since $S$ is an empty set, then can't find such $x \in S$ where $\neg Rxx$.
Hence, $R$ is not \emph{not} relexive, hence relexive.\\
\\
Also note that a relation $R \subseteq S^2$ is asymmetric if for no pair $x,y \in S$ we have both $Rxy$ and $Ryx.$
Since $S$ is empty, then we couldn't find such pair of $x,y \in S$ where $Rxy$ and $Ryx$.
Hence, $R$ is asymmetric. 
\end{proof}

\newpage
\noindent
2. Prove that the transitive closure, $R^+$ of a binary relation $R$ is actually 
transitive. You can do this however you like. Here is one suggestion:
\begin{enumerate}
\renewcommand{\labelenumi}{\alph{enumi}.}
\item Prove that, where $X$ is a set and $R \subseteq X^2$ (i.e.,\ it is a binary 
relation on $X$), then $\langle a,b \rangle \in R^n$ (as defined on page 191 of 
your text) iff there is a path of length $n$ between $a$ and $b$ in the graph of 
$R$. 
\item Use this result to then argue that if $\langle a,b\rangle \in R^+$ and 
$\langle b,c\rangle \in R^+$, then $\langle a,c\rangle \in R^+$.
\end{enumerate}
~~~
\begin{center}
    ------(a)------
\end{center}
\begin{proof}
    Let's first prove the forward direction of the iff statement. 
    Notice that by the definition of $R^n$, we have,
    \begin{align*}
        R^{n}ab &\implies \exists y_1 \in X, (R^{n-1}ay_1 \land Ryb)\\
            &\implies \exists y_2 \in X, (R^{n-2}ay_2 \land Ry_2y_1)\\
            &\quad \vdots\\
            &\implies \exists y_{n-2} \in X, (R^{2}ay_{n-2} \land Ry_{n-2}y_{n-3})\\
            &\implies \exists y_{n-1} \in X, (R^{1}ay_{n-1} \land Ry_{n-1}y_{n-2})
    \end{align*}
    And by definition, $R^{1}ay_{n-1} = Ray_{n-1}$.\\
    Hence, we have $y_1, y_2, \dots, y_{n-1} \in X$ s.t. $Ray_{n-1}, Ry_{n-1}y_{n-2}, Ry_{n-2}y_{n-3},, \dots ,Ry_2y_1, Ry_1b$.\\
    Now, we can construct path of a length $n$ from $a$ to $b$ in the graph of $R$: 
    $$a \to y_{n-1} \to y_{n-2} \to y_{n-2} \to \dots \to y_2 \to y_1 \to b$$
    \\
    Let's prove the other direction. Suppose there exists a path of length $n$ from $a$ to $b$ in the graph of $R$:
    $$a \to x_1 \to x_2 \to \dots \to x_{n-2} \to x_{n-1} \to b$$
    \\
    where $x_1, x_2, \dots, x_{n-2}, x_{n-1} \in X$ s.t. $Rax_1, Rx_1x_2, \dots, Rx_{n-2}x_{n-1}, R_{n-1}b$.\\
    \\
    Thus, we have,
    \begin{align*}
        Rax_1 &\implies R^1ax_1 \tag{by definition of $R^1$}\\
              &\implies R^2ax_2 \tag{since $R^1ax_1$ and $Rx_1x_2$}\\
              &\implies R^3ax_3 \tag{since $R^2ax_2$ and $Rx_2x_3$}\\
              &\quad \vdots\\
              &\implies R^{n-2}ax_{n-2} \tag{since $R^{n-3}ax_{n-3}$ and $Rx_{n-3}x_{n-2}$}\\
              &\implies R^{n-1}ax_{n-1} \tag{since $R^{n-2}ax_{n-2}$ and $Rx_{n-2}x_{n-1}$}\\
              &\implies R^{n}ab \tag{since $R^{n-1}ax_{n-1}$ and $Rx_{n-1}b$}
    \end{align*}
\end{proof}
~~~
\begin{center}
    ------(b)------
\end{center}
\begin{proof}
    Suppose we have $\langle a,b \rangle \in R^+$ and $\langle b,c \rangle \in R^+$. 
    Then there must exists $n_1 \in \mathbb{N}$ s.t. $\langle a, b \rangle \in R^{n_1} \subseteq R^+$, by the definition of $R^+$.
    Similarly, there must exists $n_2 \in \mathbb{N}$ s.t. $\langle b, c \rangle \in R^{n_2} \subseteq R^+$.
    Therefore, based on the results from (a), there must exists a path of length $n_1$ from $a$ to $b$ in the graph of $R$.
    And similarly, a path of length $n_2$ from $b$ to $c$. 
    This means that we would have a path of length $n_1 + n_2$ from $a$ to $c$ in the graph of $R$.
    Hence, by (a), we have $\langle a, c \rangle \in R^{n_1 + n_2}$. 
    Since $R^{n_1 + n_2} \subseteq R^+$, we arrive at $\langle a, c \rangle \in R^+$.
\end{proof}


\newpage
\noindent
3. Let $R \subseteq X^2$ be a binary relation on $X$ that is antisymmetric, 
transitive and total/connex (notice that totality implies reflexivity). Show that 
$$R' := \{\langle x,y \rangle \in X^2: \langle y, x \rangle \notin R\}$$
\noindent
is irreflexive, transitive, and connected/semi-connex. (A relation on $X$  is \emph{connected}, 
or \emph{semi-connex}, if, for all $x,y \in X$, if $x\neq y$ then 
either $\langle x, y \rangle \in R$ or $\langle y, x \rangle \in R$. A relation 
is \emph{connex}, or \emph{total}, when, for all $x,y \in X$, either $\langle x, 
y \rangle \in R$ or $\langle y, x \rangle \in R$.)
\begin{center}
    $\ast$~$\ast$~$\ast$
\end{center}
\begin{proof}
    First note that either $X = \varnothing$ or $X \neq \varnothing$.\\
    \\
    \textbf{CASE 1 }($X = \varnothing$):\\
    \\
    If $X = \varnothing$, then $\varnothing = R \subseteq X^2 = \varnothing$. 
    Observe that the assumption on $R$ being antisymmetric, transitive and total still holds.\\
    
    \indent 1) $R$ is vacuously antisymmetric. Note that $R \subseteq X^2$ is antisymmetric iff
    $\forall x, y \in X, ([Rxy \wedge Ryx] \implies x = y)$. Therefore, $R$ is \emph{not} antisymmetric 
    iff $\exists x, y \in X, (Rxy \wedge Ryx \wedge x \neq y)$. Since $R$ is empty, then
    we can't find a pair of $x, y$ in $R$ that satisfy the requirement. $R$ is not \emph{not} 
    antisymmetric, hence antisymmetric.\\

    \indent 2) $R$ is vacuously transitive. Note that $R$ is \emph{not} transitive iff 
    $\exists x,y,z \in X, (Rxy \wedge Ryz \wedge \neg Rxz)$. Since $R$ is empty, then we can't find
    the corresponding pairs in $x,y,z$ that can satisfy the requirement. 
    $R$ is not \emph{not} transitive, hence transitive.\\
    
    \indent 3) $R$ is vacuously total. Note that $R$ is \emph{not} total iff
    $\exists x, y \in X, (\neg Rxy \wedge \neg Ryx)$. Since $X$ is empty, we can't find the
    corresponding pairs of $x, y$ that can satisfy the requirement. $R$ is not \emph{not} 
    total, hence total.\\

    \noindent We now show that $R'$ is irreflexive, transitive, and connected.
    First note that $\varnothing = R' = \{\langle x,y \rangle \in X^2 = \varnothing: \langle y, x \rangle \notin R\}$.\\

    \indent 1) \underline{$R'$ is vacuously irreflexive.} Note that $R'$ is irreflexive iff 
    $\forall x \in X, R'xx$. Then $R'$ is vacuously irreflexive as $X$ is empty.
    Or more percisely, Note that $R'$ is \emph{not} irreflexive iff 
    $\exists x \in X, \neg R'xx$. But since $X$ is empty, then we can't find such $x \in X$
    that fits the description, which makes $R'$ not \emph{not} irreflexive, hence irreflexive.\\

    \indent 2) \underline{$R'$ is vacuously transitive.} It's similar as above but here's another argument.
    Note that $R'$ is trantive iff $\forall x,y,z \in X, ([R'xy \wedge R'yz] \implies R'yz)$.
    But since $R'$ is empty, the antecedent of the material conditional statement $R'xy \wedge R'yz$
    is always false. This makes the universal statement always true, hence $R'$ is transitive.\\

    \indent 3) \underline{$R'$ is vacuously connected.} Note that $R'$ is connected iff
    $\forall x, y \in X, (x \neq y \implies [R'xy \lor R'yx])$. But since $X$ is empty, the antecedent 
    of the material conditional statement $x \neq y$ can't be satisfied. This makes the universal 
    statement always true, hence $R'$ is connected.\\
    \\
    \textbf{CASE 2 }($X \neq \varnothing$):\\
    \\
    If $X$ is nonempty, then either $\varnothing = R \subseteq X^2$ or $R \neq \varnothing$.
    However, note that $R$ is total; thus, $\forall x \in X, \langle x,x \rangle \in R$.
    Hence, it's necessary that $R \neq \varnothing$.\\
    \\
    \indent 1) \underline{$R'$ is irreflexive.}
    Pick any $x \in X$. We have $Rxx$ as shown before.
    Therefore, $\neg R'xx$ according to the definition of $R'$. Hence, $R'$ is irreflexive.\\

    \indent 2) \underline{$R'$ is transitive.}
    Suppose we have $R'xy$ and $R'yz$ for any $x,y,z \in X$. 
    Notice that since we've etablished that $R'$ is irreflexive, $x \neq y$ and $y \neq z$; 
    otherwise, we would have $R'xx$ or $R'zz$, which would cause contradictions.
    From $R'yx$, we would have $\neg Ryx$, and similarly $\neg Rzy$ from $R'yz$. 
    Note that $R$ is total, meaning that there must a tuple in $R$ for any two elements in $X$. 
    Therefore, if $\neg Ryx$, then $Rxy$, and similiarly, if $\neg Rzy$, then $Ryz$.
    Also note that $R$ is transitive. Therefore, we arrive at $Rxz$ as we have both $Rxy$ and $Ryz$.\\
    \\
    Now, we have two cases. Either $x \neq z$ or $x = z$.\\
    \\
    CASE 2.1 ($x \neq z$): 
    Since $R$ is also antisymmetric, then if $x \neq z$, then either $\neg Rxz$ or $\neg Rzx$. Since $Rxz$, then it's necessary that $\neg Rzx$.
    Therefore, we arrive at $R'xz$ based on the definition of $R'$. $R'$ is transitive in 
    this case.\\
    \\
    CASE 2.2 ($x = z$): 
    Since we have both $Rxy$ and $Ryz$, if $x = z$, then we instead have both $Rxy$ and $Ryx$. 
    However, note that $R$ is antisymmetric; therefore, $x = y$.
    This is a contradiction because it's necessary that $x \neq y$.
    We have now violated our first set of assumptions. 
    Hence, it's not possible that $x = z$ and we can dismiss this case.\\
    \\
    Thus, $\forall x,y,z \in X, ([R'xy \land R'yz] \implies R'zy)$, meaning $R'$ is transitive.\\
    \\
    Also notice that in order to have both $R'xy$ and $R'yz$, it's necessary that $x \neq y, y \neq z$ and $x \neq z$ according to previous discussions.
    However, if we could't find such $x,y,z$ s.t. $(R'xy \land R'yz)$, like when $|X|<3$, then the above antecedent $(R'xy \wedge R'yz)$ would be always false.
    Therefore, $R'$ is vacuously transitive in this case.\\

    \indent 3) \underline{$R'$ is connected.}
    Pick any $x,y \in X$ s.t. $x \neq y$. 
    Since $R$ is antisymmetric, then either $\neg Rxy$ or (inclusive) $\neg Ryx$.
    But since $R$ is also total, then either $Rxy$ or (inclusive) $Ryx$.
    Thus taking together, we have either $(\neg Rxy \land Ryx)$ or (exclusive) $(Rxy \land \neg Ryx)$.
    If it's $(\neg Rxy \land Ryx)$, then by definition, we have $R'xy$.
    Similarly, if it's $(Rxy \land \neg Ryx)$, then $R'yx$.
    Thus, $\forall x, y \in X, (x \neq y \implies R'xy \lor R'yx)$.
    $R'$ is connected.\\
    \\
    Similarly, if we couldn't find a pair of $x, y \in X$ s.t. $x \neq y$, like when $|X| = 1,$ then the antecedent of the mateiral condition is always false.
    $R'$ is vacously connected in this case.



\end{proof}

\newpage
\noindent
4. Call a relation $R$ on a set $S$ \emph{universal} when, for all $x, y \in S$, 
$\langle x, y \rangle \in R$. Let $R$ be an equivalence relation and define, for each
$x \in S$, the set 
$$[x] = \{y \in S : \langle x , y \rangle \in R\}$$
\noindent
Call this the \emph{equivalence class} of $x$ (with respect to $R$). Prove that all
of the following are true:
\begin{enumerate}
\renewcommand{\labelenumi}{\alph{enumi}.}
\item $x \in [x]$, for all $x \in S$;
\item $R$ is universal on each equivalence class $[x]$;
\item Every element of $S$ is in one and only one equivalence class.
\end{enumerate}
~~~
\begin{center}
    ------(a)------
\end{center}
\begin{proof}
    Pick any $x \in S$ and let $[x]_R$ be the equivalence class defined above. 
    Given $R$ is an equivalence relation, we have $\langle x,x \rangle \in R$ as $R$ is relexive.
    Hence, $x \in [x]_R$.
\end{proof}
~~~
\begin{center}
    ------(b)------
\end{center}
\begin{proof}
    Suppose we have a equivalence class $[x]_R$ as defined above. And suppose $p, q \in [x]_R$.
    Since $p \in [x]_R,$ we have $\langle x, p \rangle \in R$. Since $q \in [x]_R$, we also have
    $\langle x,q \rangle \in R$. Note that $R$ is symmetric as it is an equivalence relation;
    therefore, we have $\langle p,x \rangle \in R$. Also note that $R$ is transitive as it is an
    equivalence relation; therefore, $\langle p,q \rangle \in R$. Hence, $\forall x_1, x_2 \in [x]_R, 
    \langle x_1, x_2 \rangle \in R$; $R$ is univeral on this equivalence class. Since we picked a
    random equivalence class, $R$ is univeral on \emph{each} $[x]_R$.
\end{proof}
~~~
\begin{center}
    ------(c)------
\end{center}
\begin{proof}
    Pick any $x \in S$. From $(a)$ we know that element $x$ is in at least one equivalence class, 
    namely $[x]_R$. We now show that $[x]_R$ is the only equivalence class with respect to 
    $R$ that $x$ is in.\\
    We know that $x \in [x]_R$ from $(a)$. But now suppose that $x \in [\bar{x}]_R$ with
    $\bar{x} \in S$. We show that $[x]_R = [\bar{x}]_R$.\\
    Pick any $s \in [x]_R$, then we know that $\langle x,s \rangle \in R$ according to the
    definition of equivalence class. Similarly, since $x \in [\bar{x}]_R$, we have 
    $\langle \bar{x}, x \rangle \in R$. Thus, we have $\langle \bar{x}, s \rangle \in R$ 
    as $R$ is transitive.
    Hence, $s \in [\bar{x}]_R$. We have $[x]_R \subseteq [\bar{x}]_R$.\\
    Now pick any $s \in [\bar{x}]_R$. From (b) we know that $R$ is univeral on $[\bar{x}]_R$ and 
    $s, x \in [\bar{x}]_R$. Thus, $\langle x, s \rangle \in R$, which means that 
    $s \in [x]_R$. We have $[x]_R \supseteq [\bar{x}]_R$. \\
    Therefore, $[x]_R = [\bar{x}]_R$. This means that any other equivalence classes $x$ is in
    are just $[x]_R$. Hence, every element of $S$ is in one and only one equivalence class.

\end{proof}

\newpage
\noindent
5. Prove that the set $\{\neg, \land, \lor\}$ is truth-functionally complete. Hint:
Consider the following truth table for some arbitrary binary connective $\circ$:
\medskip
\begin{center}
\begin{tabular}{cc|c}
$\varphi$ & $\psi$ & $\varphi \circ \psi$ \\ \hline
$\mathsf{t}$ & $\mathsf{t}$ & $\mathsf{v}_1$ \\
$\mathsf{t}$ & $\mathsf{f}$ & $\mathsf{v}_2$ \\
$\mathsf{f}$ & $\mathsf{t}$ & $\mathsf{v}_3$ \\
$\mathsf{f}$ & $\mathsf{f}$ & $\mathsf{v}_4$ \\
\end{tabular}
\end{center}
\medskip
\noindent
where $\mathsf{v}_i$ is the truth value for that row. There are at least two 
options now.  One is a brute force argument that is effective, but not so exciting.
See if you can come up with a different one. Start by thinking about what you could
do if all the values are $\mathsf{f}$. What would express such a function? Then 
think about what would happen if one of the values were $\mathsf{t}$. How might you
express that? Then go from there\dots
\begin{center}
    $\ast$~$\ast$~$\ast$
\end{center}
\begin{proof}
    We need to show that given any truth function $f \colon \{\mathsf{t}, \mathsf{f}\}^{n \in \mathbb{N}} \longmapsto \{\mathsf{t}, \mathsf{f}\}$, $f$ can be described by $f_{\neg} \colon \{\mathsf{t}, \mathsf{f}\} \longmapsto \{\mathsf{t}, \mathsf{f}\} $ and $f_{\lor}, f_{\land} \colon \{\mathsf{t}, \mathsf{f}\}^2 \longmapsto \{\mathsf{t}, \mathsf{f}\}$.\\
    \\
    Suppose we have a $f \colon \{\mathsf{t}, \mathsf{f}\}^n \longmapsto \{\mathsf{t}, \mathsf{f}\}, n \in \mathbb{N}$.
    $f$ inputs a n-tuple of truth values and outputs either only $\mathsf{t}$ or only $\mathsf{f}$.
    We thus have $2^n$ possibilities for inputs and 2 possibilities for outputs. 
    Hence, in total $2^{2^n}$ possible mappings.
    And $f$ is one of the $2^{2^n}$ possible mappings, for any fixed $n$.\\
    \\
    Note that each mapping (function) can be described by a truth table.
    For each $n$, suppose the n-tuple of truth values are from $n$ distinct propositional variables $\psi_1, \psi_2, \dots, \psi_n$.
    Therefore, each truth table has $2^n$ rows.
    We have in total $2^{2^n}$ different truth tables.
    And each row is a tuple of truth values given by valuation function $v$ and the output of $f$,
    $$\langle v_1(\psi_1), v_2(\psi_2), \dots, v_n(\psi_n), f \rangle$$
    \\
    We need to construct a WFF for each truth table, using $\psi_1, \psi_2, \dots, \psi_n$ with only $\neg, \land, \lor$ as connectives, that has the same output as $f$ given the same input.
    So that each $f$ can be written as a compositions of $f_{\neg}, f_{\lor}, f_{\land}$.\\
    \\
    We first show that we can construct a WFF for each truth table that represents one of $4 = 2^{2^1}$ possible mappings when $n = 1$.
    Here there are,
    \begin{center}
        \begin{tabular}{c|c}
            $\varphi$ & $\varphi \lor \neg \varphi$\\ \hline
            $\mathsf{t}$ & $\mathsf{t}$\\
            $\mathsf{f}$ & $\mathsf{t}$\\
        \end{tabular}
        ~~~
        \begin{tabular}{c|c}
            $\varphi$ & $\varphi$ \\ \hline
            $\mathsf{t}$ & $\mathsf{t}$\\
            $\mathsf{f}$ & $\mathsf{f}$\\
        \end{tabular}
        ~~~
        \begin{tabular}{c|c}
            $\varphi$ & $\neg \varphi$\\ \hline
            $\mathsf{t}$ & $\mathsf{f}$\\
            $\mathsf{f}$ & $\mathsf{t}$\\
        \end{tabular}
        ~~~
        \begin{tabular}{c|c}
            $\varphi$ & $\varphi \land \neg \varphi$\\ \hline
            $\mathsf{t}$ & $\mathsf{f}$\\
            $\mathsf{f}$ & $\mathsf{f}$\\
        \end{tabular}
    \end{center}
    Thus, suppose $f$ can be captured by the first table, then we know 
    $$f(x) = f_{\lor}(x, f_{\neg}(x)), x \in \{\mathsf{t},\mathsf{f}\}$$
    \\
    Then for each $n \geq 2$, we construct a WFF for each truth table using the following algorithm:\\
    \\
    We first look for tuples where $f = \mathsf{t}$.
    For each such tuple, we first make each propositional variable a conjunct,
    $$(\psi_1 \land \psi_2 \land \dots \land \psi_n)$$
    If $v(\psi_i) = \mathsf{f}, i \leq n$, then we put $\neg$ in front of it.
    For example, if $n = 5$ and $v(\psi_2) = v(\psi_3) = \mathsf{f}$, then the conjuction would be,
    $$(\psi_1 \land \neg \psi_2 \land \neg \psi_3 \land \psi_4 \land \psi_5)$$
    We do this for all the tuples, and use $\lor$ to connect all of them.\\
    \\
    If we can't find a tuple where $f = \mathsf{t}$ (i.e. all rows are $\mathsf{f}$).
    Then we simply put the negation of the WFF for when all rows are $\mathsf{t}$.\\
    \\
    And here are all $16 = 2^{2^2}$ possible mappings when $n = 2$ and their each WFF that expresses the same truth value, using the algorithm:

    \begin{center}
        \begin{tabular}{cc|c}
            $\varphi$ & $\psi$ & $(\varphi \land \psi) \lor (\varphi \land \neg \psi) \lor (\neg \varphi \land \psi) \lor (\neg \varphi \land \neg \psi)$ \\ \hline
            $\mathsf{t}$ & $\mathsf{t}$ & $\mathsf{t}$ \\
            $\mathsf{t}$ & $\mathsf{f}$ & $\mathsf{t}$ \\
            $\mathsf{f}$ & $\mathsf{t}$ & $\mathsf{t}$ \\
            $\mathsf{f}$ & $\mathsf{f}$ & $\mathsf{t}$ \\
        \end{tabular}
        ~~
        \begin{tabular}{cc|c}
            $\varphi$ & $\psi$ & $\neg [(\varphi \land \psi) \lor (\varphi \land \neg \psi) \lor (\neg \varphi \land \psi) \lor (\neg \varphi \land \neg \psi)]$ \\ \hline
            $\mathsf{t}$ & $\mathsf{t}$ & $\mathsf{f}$ \\
            $\mathsf{t}$ & $\mathsf{f}$ & $\mathsf{f}$ \\
            $\mathsf{f}$ & $\mathsf{t}$ & $\mathsf{f}$ \\
            $\mathsf{f}$ & $\mathsf{f}$ & $\mathsf{f}$ \\
        \end{tabular}
        ~~
        \begin{tabular}{cc|c}
            $\varphi$ & $\psi$ & $(\varphi \land \psi) \lor (\varphi \land \neg \psi) \lor (\neg \varphi \land \psi)$ \\ \hline
            $\mathsf{t}$ & $\mathsf{t}$ & $\mathsf{t}$ \\
            $\mathsf{t}$ & $\mathsf{f}$ & $\mathsf{t}$ \\
            $\mathsf{f}$ & $\mathsf{t}$ & $\mathsf{t}$ \\
            $\mathsf{f}$ & $\mathsf{f}$ & $\mathsf{f}$ \\
        \end{tabular}
        ~~
        \begin{tabular}{cc|c}
            $\varphi$ & $\psi$ & $(\varphi \land \psi) \lor (\varphi \land \neg \psi) \lor (\neg \varphi \land \neg \psi)$ \\ \hline
            $\mathsf{t}$ & $\mathsf{t}$ & $\mathsf{t}$ \\
            $\mathsf{t}$ & $\mathsf{f}$ & $\mathsf{t}$ \\
            $\mathsf{f}$ & $\mathsf{t}$ & $\mathsf{f}$ \\
            $\mathsf{f}$ & $\mathsf{f}$ & $\mathsf{t}$ \\
        \end{tabular}
        ~~
        \begin{tabular}{cc|c}
            $\varphi$ & $\psi$ & $(\varphi \land \psi) \lor (\neg \varphi \land \psi) \lor (\neg \varphi \land \neg \psi)$ \\ \hline
            $\mathsf{t}$ & $\mathsf{t}$ & $\mathsf{t}$ \\
            $\mathsf{t}$ & $\mathsf{f}$ & $\mathsf{f}$ \\
            $\mathsf{f}$ & $\mathsf{t}$ & $\mathsf{t}$ \\
            $\mathsf{f}$ & $\mathsf{f}$ & $\mathsf{t}$ \\
        \end{tabular}
        ~~
        \begin{tabular}{cc|c}
            $\varphi$ & $\psi$ & $(\varphi \land \neg \psi) \lor (\neg \varphi \land \psi) \lor (\neg \varphi \land \neg \psi)$ \\ \hline
            $\mathsf{t}$ & $\mathsf{t}$ & $\mathsf{f}$ \\
            $\mathsf{t}$ & $\mathsf{f}$ & $\mathsf{t}$ \\
            $\mathsf{f}$ & $\mathsf{t}$ & $\mathsf{t}$ \\
            $\mathsf{f}$ & $\mathsf{f}$ & $\mathsf{t}$ \\
        \end{tabular}
        ~~
        \begin{tabular}{cc|c}
            $\varphi$ & $\psi$ & $(\varphi \land \psi) \lor (\neg \varphi \land \psi)$ \\ \hline
            $\mathsf{t}$ & $\mathsf{t}$ & $\mathsf{t}$ \\
            $\mathsf{t}$ & $\mathsf{f}$ & $\mathsf{f}$ \\
            $\mathsf{f}$ & $\mathsf{t}$ & $\mathsf{t}$ \\
            $\mathsf{f}$ & $\mathsf{f}$ & $\mathsf{f}$ \\
        \end{tabular}
        ~~
        \begin{tabular}{cc|c}
            $\varphi$ & $\psi$ & $(\varphi \land \psi) \lor (\varphi \land \neg \psi)$ \\ \hline
            $\mathsf{t}$ & $\mathsf{t}$ & $\mathsf{t}$ \\
            $\mathsf{t}$ & $\mathsf{f}$ & $\mathsf{t}$ \\
            $\mathsf{f}$ & $\mathsf{t}$ & $\mathsf{f}$ \\
            $\mathsf{f}$ & $\mathsf{f}$ & $\mathsf{f}$ \\
        \end{tabular}
        ~~
        \begin{tabular}{cc|c}
            $\varphi$ & $\psi$ & $(\varphi \land \psi) \lor (\neg \varphi \land \neg \psi)$ \\ \hline
            $\mathsf{t}$ & $\mathsf{t}$ & $\mathsf{t}$ \\
            $\mathsf{t}$ & $\mathsf{f}$ & $\mathsf{f}$ \\
            $\mathsf{f}$ & $\mathsf{t}$ & $\mathsf{f}$ \\
            $\mathsf{f}$ & $\mathsf{f}$ & $\mathsf{t}$ \\
        \end{tabular}
        ~~
        \begin{tabular}{cc|c}
            $\varphi$ & $\psi$ & $(\varphi \land \neg \psi) \lor (\neg \varphi \land \psi)$ \\ \hline
            $\mathsf{t}$ & $\mathsf{t}$ & $\mathsf{f}$ \\
            $\mathsf{t}$ & $\mathsf{f}$ & $\mathsf{t}$ \\
            $\mathsf{f}$ & $\mathsf{t}$ & $\mathsf{t}$ \\
            $\mathsf{f}$ & $\mathsf{f}$ & $\mathsf{f}$ \\
        \end{tabular}
        ~~
        \begin{tabular}{cc|c}
            $\varphi$ & $\psi$ & $(\varphi \land \neg \psi) \lor (\neg \varphi \land \neg \psi)$ \\ \hline
            $\mathsf{t}$ & $\mathsf{t}$ & $\mathsf{f}$ \\
            $\mathsf{t}$ & $\mathsf{f}$ & $\mathsf{t}$ \\
            $\mathsf{f}$ & $\mathsf{t}$ & $\mathsf{f}$ \\
            $\mathsf{f}$ & $\mathsf{f}$ & $\mathsf{t}$ \\
        \end{tabular}
        ~~
        \begin{tabular}{cc|c}
            $\varphi$ & $\psi$ & $(\neg \varphi \land \psi) \lor (\neg \varphi \land \neg \psi)$ \\ \hline
            $\mathsf{t}$ & $\mathsf{t}$ & $\mathsf{f}$ \\
            $\mathsf{t}$ & $\mathsf{f}$ & $\mathsf{f}$ \\
            $\mathsf{f}$ & $\mathsf{t}$ & $\mathsf{t}$ \\
            $\mathsf{f}$ & $\mathsf{f}$ & $\mathsf{t}$ \\
        \end{tabular}
        ~~~~~~~~~~~~~
        \begin{tabular}{cc|c}
            $\varphi$ & $\psi$ & $\varphi \land \psi$ \\ \hline
            $\mathsf{t}$ & $\mathsf{t}$ & $\mathsf{t}$ \\
            $\mathsf{t}$ & $\mathsf{f}$ & $\mathsf{f}$ \\
            $\mathsf{f}$ & $\mathsf{t}$ & $\mathsf{f}$ \\
            $\mathsf{f}$ & $\mathsf{f}$ & $\mathsf{f}$ \\
        \end{tabular}
        ~~
        \begin{tabular}{cc|c}
            $\varphi$ & $\psi$ & $\varphi \land \neg \psi$ \\ \hline
            $\mathsf{t}$ & $\mathsf{t}$ & $\mathsf{f}$ \\
            $\mathsf{t}$ & $\mathsf{f}$ & $\mathsf{t}$ \\
            $\mathsf{f}$ & $\mathsf{t}$ & $\mathsf{f}$ \\
            $\mathsf{f}$ & $\mathsf{f}$ & $\mathsf{f}$ \\
        \end{tabular}
        ~~
        \begin{tabular}{cc|c}
            $\varphi$ & $\psi$ & $\neg \varphi \land \psi$ \\ \hline
            $\mathsf{t}$ & $\mathsf{t}$ & $\mathsf{f}$ \\
            $\mathsf{t}$ & $\mathsf{f}$ & $\mathsf{f}$ \\
            $\mathsf{f}$ & $\mathsf{t}$ & $\mathsf{t}$ \\
            $\mathsf{f}$ & $\mathsf{f}$ & $\mathsf{f}$ \\
        \end{tabular}
        ~~
        \begin{tabular}{cc|c}
            $\varphi$ & $\psi$ & $\neg \varphi \land \neg \psi$ \\ \hline
            $\mathsf{t}$ & $\mathsf{t}$ & $\mathsf{f}$ \\
            $\mathsf{t}$ & $\mathsf{f}$ & $\mathsf{f}$ \\
            $\mathsf{f}$ & $\mathsf{t}$ & $\mathsf{f}$ \\
            $\mathsf{f}$ & $\mathsf{f}$ & $\mathsf{t}$ \\
        \end{tabular}
    \end{center}
$$$$
Thus, suppose $f$ is the first table in the second last row, then 
$$f(x,y) = f_{\lor}(f_{\land}(x,f_{\neg}(y)),f_{\land}(f_{\neg}(x),f_{\neg}(y))), \; x, y \in \{\mathsf{t}, \mathsf{f}\}$$
The algorithm works because each long conjuction of the resulting WFF will only be true with the inputs from its tuple.
Since we also then connect them with disjunction, we have made sure that the resulting WFF will only be true whenever $f$ is true.
This preserves the mapping of each $f$.\\
\\
Hence, for any truth function $f$, $f$ can be expressed by a composition of $f_{\neg}, f_{\land}, f_{\lor}$.
The set $\{\neg, \land, \lor\}$ is truth-functionally complete.

\end{proof}
\end{document}